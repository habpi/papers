\documentclass[journal]{new-aiaa}
%\documentclass[conf]{new-aiaa} for conference papers
\usepackage[utf8]{inputenc}

\usepackage{graphicx}
\usepackage{amsmath}
\usepackage[version=4]{mhchem}
\usepackage{siunitx}
\usepackage{longtable,tabularx}
\setlength\LTleft{0pt} 

\title{HabPi: An Opensource Extensible Framework for High Altitude
Balloon Data Collection}

\author{Robert Lowe\footnote{Assistant Professor, Division of
Mathematics and Computer Science, Maryville College, 502 East Lamar
Alexander Parkway, Maryville TN 37804}}
\affil{Maryville College}

\begin{document}

\maketitle

\begin{abstract}
One of the obstacles facing beginning high altitude balloonists
is the cost of the myriad sensors required to acquire usable data.
In addition to this problem, aligning sensor data after a flight can
prove tricky.  Finally, there is a harsh reality we all face.  Trees
exist, and payload stacks are attracted to trees!

HabPi is an open source project which attempts to address these
problems.  This is a sensor array which is based on a raspberry pi,
a sense hat, and several one-wire sensors.  The sensors are all
synchronized, so data registration is not a problem.  HabPi also acts
as a wifi access point which makes data retrieval possible even when
the stack is trapped in a tree (or is otherwise inaccessible).  HabPi
is a simple extensible framework which allows users to add additional
experiments beyond the default setup.  The default setup provides
temperature, pressure, magnetometer, gyroscope, accelerometer, and
video imaging all for less than \$200.00.
\end{abstract}

\section{Introduction}
\subsection{Motivation}
\subsection{Educational Use}
\subsection{Organization of this Paper}

\section{Construction}
\subsection{Parts List}
\subsection{Box Dimensions}
\subsection{Wiring the Sense Hat}
\subsection{Wiring the 1-Wire Sensor Array}

\section{Habpi Software}
\subsection{Overview}
\subsection{Raspberry Pi Setup}
\subsection{Setting up HabPi}
\subsection{Creating Experiments}
\subsection{HabPi User Interface}
\subsubsection{Sense HAT Joystick}
\subsubsection{Built-In webserver}

\section{Performance and Flight Data}
\subsection{Sense HAT Challenges}
\subsection{Temperature Data}
\subsubsection{Sense HAT}
\subsubsection{DS18B20}
\subsection{Other Sensors}

\end{document}
